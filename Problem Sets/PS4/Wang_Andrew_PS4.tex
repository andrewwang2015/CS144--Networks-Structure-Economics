\documentclass[12 pt]{article}
\usepackage{fancyhdr}
\usepackage[margin = 1 in]{geometry}
\usepackage{amsmath}
\usepackage{enumerate}
% \usepackage{indentfirst}
\pagestyle{fancy}
\usepackage{graphicx}
\usepackage[version=3]{mhchem}
\fancyhf{}
\usepackage{sectsty}	
\lhead{Andrew Wang}
\chead{\textbf{CS144}}
\rhead{Wierman}
%\sectionfont{\fontsize{15}{18}\selectfont}
\usepackage{graphicx}
\usepackage{array}
\usepackage{listings}
\newcolumntype{P}[1]{>{\centering\arraybackslash}p{#1}}
\newcolumntype{M}[1]{>{\centering\arraybackslash}m{#1}}

\begin{document}
	\begin{center}
		\section*{Problem Set 4}
	\end{center}
	
	\noindent Collaborated with: Steven Brotz, David Kawashima, Jessica Choi. \\
	\noindent Used one (1) late token. \\
	
	\subsection*{Problem 1}
	\noindent \textbf{a.} Calculating the stationary distribution using a system of equations
	and Wolfram, we get:
	
	\[\pi = (\frac{7}{22}, \frac{7}{22}, \frac{4}{11})\]
	
	\noindent If we take $P^n$ as $n \rightarrow \infty$, we see that it approaches: 

	\[\begin{bmatrix}
	\frac{7}{22} & \frac{7}{22} & \frac{4}{11} \\
	\frac{7}{22} & \frac{7}{22} &  \frac{4}{11} \\
	\frac{7}{22} & \frac{7}{22} &  \frac{4}{11}
	\end{bmatrix}
	\]
	
	\noindent Thus, because each row in $P^n$ for large $n$ is the same, and we know that $\sum \pi (i) = 1$, and thus $\sum \pi_0 (i) = 1$, for any starting $\pi_0$, we have that: 
	
	\[\lim_{n \rightarrow \infty} \pi_0 P^n = \pi = (\frac{7}{22}, \frac{7}{22}, \frac{4}{11})\]
	
	\noindent Thus, $\pi$ does not depend on $\pi_0$, meaning that our stationary distribution does not depend on our starting distribution at all. In other words, the stationary distribution represents the steady state/ equilibrium. \\
	
	\noindent \textbf{b.} Calculating the stationary distribution using a system of equations
	and Wolfram, we get:
	
	\[\pi = (\frac{7}{16}, \frac{1}{8}, \frac{1}{16}, \frac{3}{8})\]
	
	\noindent If we take $P^n$ as $n \rightarrow \infty$, we see that it approaches: 
	
	\[\begin{bmatrix}
	0 & 0.25 & 0 & 0.75 \\
	0.875 & 0 & 0.125 & 0 \\
	0 & 0.25 & 0 & 0.75 \\
	0.875 & 0 & 0.125 & 0
	\end{bmatrix}
	\]
	
	\noindent From this, we can see that because not all of the rows are equivalent, that $\lim_{n\rightarrow \infty} \pi_0 P^n$ does not converge. \\
	
	\newpage
	
	\subsection*{Problem 4}
	\noindent \textbf{a.}
	\begin{lstlisting}
# CommonFriends 

function main(){
	# Define the input
	listOfFriendships = list of (Person, [List of Friends])
	
	# Run MapReduce on the list listOfFriendship
	listOfOutputs = MapReduce(listOfFriendship)
	
	# Output result
	Print contents of listOfOutputs on console.
}

function MapReduce(listOfFriendships){

	# Map step:
	For each(key, value) pair in listOfFriendship, run Map(key, value)
	
	# Collector step:
	Collect all outputs from Map() in the form of (key, value)
		pairs in a new list called listOfMapOutputs
	Make a list of distinct keys in listOfMapOutputs
	For each unique key, concatenate the corresponding value lists
	
	For each such (key, listOfValues), run Reduce() on them.
	Concatenate the results from all calls to Reduce() into a 
		list and return it.

}

function Map(person, listOfFriends){
	# key = person, values = listOfFriends for that person
	
	For each friend in listOfFriends{
		# Sort so that pairs of people get emitted in
		# lexicographical order
		EmitInteremediate(sorted(person, friend), 
			lstOfFriends)
		}
	}
}

function Reduce(pair, listValues){
	
	# For input ((A,B), [lst1, lst2]),
	# To get mutual friends, we simply
	# take the intersection of lst1
	# and lst2 to get mutual friends of
	# A and B
	
	mutual_friends = []
	lst1 = listValues[0]
	lst2 = listValues[1]
	for friend in lst1{
		if friend in lst2 {
			add friend to mutual_friends
		}
	}
	return (pair, mutual_friends)
}
	\end{lstlisting}
	
	\noindent \textbf{b.}
	\begin{lstlisting}
	
# High school days 

function main(){
	# Define the input
	listOfFilenames = list of filenames containing test scores
		of all students
	listKeyValues[] = list of pairs (filename, fileScores)
		where fileScores is a string containing the scores,
		separated by spaces, of the file with name `filename'
	
	
	# Run MapReduce on the list listOfFilenames
	listOfOutputs = MapReduce(listKeyValues)

	# Output result
	Print contents of listOfOutputs on console.
}

function MapReduce(listKeyValues){

	# Map step:
	For each(key, value) pair in listKeyValues, run Map(key, value)

	# Collector step:
	Collect all outputs from Map() in the form of (key, value)
	pairs in a new list called listOfMapOutputs.
	Make a list of distinct keys in listOfMapOutputs.
	For each unique key, create a list of the values 
	to create (key, listOfValues) pairs.
	
	# Reduce step:
	For each such (key, listOfValues), run Reduce() on them.
	
	# Return the results
	Concatenate the results of all calls to Reduce() into a list.
	Sort this list in reverse (decreasing) order.
	Output this list. 
}

function Map(filename, fileScores){
	# key = filename, values = scores in that filename
	
	# Assuming scores are separated by spaces
	score[] = Split fileScores by " ".
	
	for each score in scores {
		EmitIntermediate(score, 1)
	}
}

function Reduce(score, listValues){
	# key = word and value = list of 1's
	# no. of ones for each score is the 
	# no. of occurrences of the word
	# in all files.
	return ([score * len(listValues)])
	
	# Return: list consisting of that score repeated
	# the number of times as the no. of occurrences
	# in all the files
}
	\end{lstlisting}
	
	\noindent \textbf{c.} The logic is as follows. Notice that if we take the unit circle centered at (0,0), it has an area equal to $\pi$. If we look at the square with corners at (-1,-1), (-1, 1), (1, -1), and (1, 1), such a square has side length = 2, so this square will have an area of 4. If we generate two numbers from our random generator (which generates numbers from [-1, +1]) and treat those two numbers as (x, y) coordinates, we are guaranteed that this point will lie in the square described above. Now to calculate $\pi$, we look at the proportion of points that not only lie in the square, but also lie in the circle. Then, if we divide the number of points that lie in the circle by the total number of points generated, it should come out to around $\frac{\pi}{4}$. Thus, if we multiply this proportion of points that lie in the circle by 4, we should be able to approximate $\pi$. \\

	\begin{lstlisting}
	
# Good old pi

function main(){
	# Define the input
	# Generate n (in this case we just say 1000)pairs of points
	let n = 1000
	lstOfPoints = []
	for i from 1 to n{
		let x = rand()
		let y = rand()
		Add (x, y) to lstOfPoints
	}
	
	# Run MapReduce on this list of points
	listOfOutputs = MapReduce(lstOfPoints)
	
	# Output result
	Print contents of listOfOutputs on console.
}

function MapReduce(lstOfPoints){

	# Map step:
	For each point (x, y) in lstOfPoints, run Map((x,y))
	
	# Collector step:
	Collect all outputs from Map() in the form of (value)
	and concatenate all values into a new list called 
	listOfMapOutputs.

	# Taking the sum of listOfMapOutputs gives us the 
	# number of points lying in the circle. Dividing
	# this by n gives us proportion of all points that 
	# lie within the circle
	
	let sum = listOfMapOutputs
	let proportion = sum / n
	
	# Multiply by 4 because as explained before,
	# the total square area is 4.
	
	let pi_approx = proportion * 4
	
	return pi_approx
}

function Map(point){
	# input = a single (x,y) point where
	# x and y are both between -1 and 1
	
	# Test if in circle. If in circle, we
	# return 1, else return 0
	
	If x^2 + y^2 <= 1 {
		EmitIntermediate(1)
	}
	else
	{
		EmitIntermediate(0)
	}

}
	
	\end{lstlisting}
	
	\noindent \textbf{d.} 
	
	\begin{lstlisting}
# GaugeTheDistance 

function main(){
	G[] = Adjacency list of the graph.
	# G[i] is a list of (neighbors, distance) tuple of node i.
	n = length(G)
	dist[] = list that will contain the distances
	from node 1 eventually. Initialize it arbitrarily.
	distUpdated[] = list that will contain distances from
	node 1 after each run of MapReduce in the
	following loop. Initialize it with $(0, \infty, \infty, \ldots, \infty)$.
	
	while( NOT stoppingCriterion(dist, distUpdated)){
		dist = distUpdated
		inputs = ((i, dist[i]), G[i]) for all i
		distUpdated = MapReduce(inputs).
	}
	print the list dist[].
}
function stoppingCriterion(dist1, dist2){
	return dist1 == dist2
}

function Map((i, dist[i]), G[i]){
	# Input is a node and its distance from node 1
	# and its adjacency list. We output a tuple
	# containing (node, distance to i) based on its
	# connection/distance to the input node
	
	EmitIntermediate(i, dist[i])
	if (dist[i] != infty)
	{
		for each (neighbor, distance) in dist[i]
		{
			EmitIntermediate(neighbor, distance + dist[i])
		}
	}
}

function Reduce(i, distancesFromi){
	# each node will have a list of distances,
	# take the minimum one
	return (i, min(distancesFromi))
}
	\end{lstlisting}
	
	
\end{document}